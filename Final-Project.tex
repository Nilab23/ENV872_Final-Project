% Options for packages loaded elsewhere
\PassOptionsToPackage{unicode}{hyperref}
\PassOptionsToPackage{hyphens}{url}
%
\documentclass[
]{article}
\usepackage{amsmath,amssymb}
\usepackage{iftex}
\ifPDFTeX
  \usepackage[T1]{fontenc}
  \usepackage[utf8]{inputenc}
  \usepackage{textcomp} % provide euro and other symbols
\else % if luatex or xetex
  \usepackage{unicode-math} % this also loads fontspec
  \defaultfontfeatures{Scale=MatchLowercase}
  \defaultfontfeatures[\rmfamily]{Ligatures=TeX,Scale=1}
\fi
\usepackage{lmodern}
\ifPDFTeX\else
  % xetex/luatex font selection
\fi
% Use upquote if available, for straight quotes in verbatim environments
\IfFileExists{upquote.sty}{\usepackage{upquote}}{}
\IfFileExists{microtype.sty}{% use microtype if available
  \usepackage[]{microtype}
  \UseMicrotypeSet[protrusion]{basicmath} % disable protrusion for tt fonts
}{}
\makeatletter
\@ifundefined{KOMAClassName}{% if non-KOMA class
  \IfFileExists{parskip.sty}{%
    \usepackage{parskip}
  }{% else
    \setlength{\parindent}{0pt}
    \setlength{\parskip}{6pt plus 2pt minus 1pt}}
}{% if KOMA class
  \KOMAoptions{parskip=half}}
\makeatother
\usepackage{xcolor}
\usepackage[margin=1in]{geometry}
\usepackage{color}
\usepackage{fancyvrb}
\newcommand{\VerbBar}{|}
\newcommand{\VERB}{\Verb[commandchars=\\\{\}]}
\DefineVerbatimEnvironment{Highlighting}{Verbatim}{commandchars=\\\{\}}
% Add ',fontsize=\small' for more characters per line
\usepackage{framed}
\definecolor{shadecolor}{RGB}{248,248,248}
\newenvironment{Shaded}{\begin{snugshade}}{\end{snugshade}}
\newcommand{\AlertTok}[1]{\textcolor[rgb]{0.94,0.16,0.16}{#1}}
\newcommand{\AnnotationTok}[1]{\textcolor[rgb]{0.56,0.35,0.01}{\textbf{\textit{#1}}}}
\newcommand{\AttributeTok}[1]{\textcolor[rgb]{0.13,0.29,0.53}{#1}}
\newcommand{\BaseNTok}[1]{\textcolor[rgb]{0.00,0.00,0.81}{#1}}
\newcommand{\BuiltInTok}[1]{#1}
\newcommand{\CharTok}[1]{\textcolor[rgb]{0.31,0.60,0.02}{#1}}
\newcommand{\CommentTok}[1]{\textcolor[rgb]{0.56,0.35,0.01}{\textit{#1}}}
\newcommand{\CommentVarTok}[1]{\textcolor[rgb]{0.56,0.35,0.01}{\textbf{\textit{#1}}}}
\newcommand{\ConstantTok}[1]{\textcolor[rgb]{0.56,0.35,0.01}{#1}}
\newcommand{\ControlFlowTok}[1]{\textcolor[rgb]{0.13,0.29,0.53}{\textbf{#1}}}
\newcommand{\DataTypeTok}[1]{\textcolor[rgb]{0.13,0.29,0.53}{#1}}
\newcommand{\DecValTok}[1]{\textcolor[rgb]{0.00,0.00,0.81}{#1}}
\newcommand{\DocumentationTok}[1]{\textcolor[rgb]{0.56,0.35,0.01}{\textbf{\textit{#1}}}}
\newcommand{\ErrorTok}[1]{\textcolor[rgb]{0.64,0.00,0.00}{\textbf{#1}}}
\newcommand{\ExtensionTok}[1]{#1}
\newcommand{\FloatTok}[1]{\textcolor[rgb]{0.00,0.00,0.81}{#1}}
\newcommand{\FunctionTok}[1]{\textcolor[rgb]{0.13,0.29,0.53}{\textbf{#1}}}
\newcommand{\ImportTok}[1]{#1}
\newcommand{\InformationTok}[1]{\textcolor[rgb]{0.56,0.35,0.01}{\textbf{\textit{#1}}}}
\newcommand{\KeywordTok}[1]{\textcolor[rgb]{0.13,0.29,0.53}{\textbf{#1}}}
\newcommand{\NormalTok}[1]{#1}
\newcommand{\OperatorTok}[1]{\textcolor[rgb]{0.81,0.36,0.00}{\textbf{#1}}}
\newcommand{\OtherTok}[1]{\textcolor[rgb]{0.56,0.35,0.01}{#1}}
\newcommand{\PreprocessorTok}[1]{\textcolor[rgb]{0.56,0.35,0.01}{\textit{#1}}}
\newcommand{\RegionMarkerTok}[1]{#1}
\newcommand{\SpecialCharTok}[1]{\textcolor[rgb]{0.81,0.36,0.00}{\textbf{#1}}}
\newcommand{\SpecialStringTok}[1]{\textcolor[rgb]{0.31,0.60,0.02}{#1}}
\newcommand{\StringTok}[1]{\textcolor[rgb]{0.31,0.60,0.02}{#1}}
\newcommand{\VariableTok}[1]{\textcolor[rgb]{0.00,0.00,0.00}{#1}}
\newcommand{\VerbatimStringTok}[1]{\textcolor[rgb]{0.31,0.60,0.02}{#1}}
\newcommand{\WarningTok}[1]{\textcolor[rgb]{0.56,0.35,0.01}{\textbf{\textit{#1}}}}
\usepackage{longtable,booktabs,array}
\usepackage{calc} % for calculating minipage widths
% Correct order of tables after \paragraph or \subparagraph
\usepackage{etoolbox}
\makeatletter
\patchcmd\longtable{\par}{\if@noskipsec\mbox{}\fi\par}{}{}
\makeatother
% Allow footnotes in longtable head/foot
\IfFileExists{footnotehyper.sty}{\usepackage{footnotehyper}}{\usepackage{footnote}}
\makesavenoteenv{longtable}
\usepackage{graphicx}
\makeatletter
\def\maxwidth{\ifdim\Gin@nat@width>\linewidth\linewidth\else\Gin@nat@width\fi}
\def\maxheight{\ifdim\Gin@nat@height>\textheight\textheight\else\Gin@nat@height\fi}
\makeatother
% Scale images if necessary, so that they will not overflow the page
% margins by default, and it is still possible to overwrite the defaults
% using explicit options in \includegraphics[width, height, ...]{}
\setkeys{Gin}{width=\maxwidth,height=\maxheight,keepaspectratio}
% Set default figure placement to htbp
\makeatletter
\def\fps@figure{htbp}
\makeatother
\setlength{\emergencystretch}{3em} % prevent overfull lines
\providecommand{\tightlist}{%
  \setlength{\itemsep}{0pt}\setlength{\parskip}{0pt}}
\setcounter{secnumdepth}{5}
\usepackage{booktabs}
\usepackage{longtable}
\usepackage{array}
\usepackage{multirow}
\usepackage{wrapfig}
\usepackage{float}
\usepackage{colortbl}
\usepackage{pdflscape}
\usepackage{tabu}
\usepackage{threeparttable}
\usepackage{threeparttablex}
\usepackage[normalem]{ulem}
\usepackage{makecell}
\usepackage{xcolor}
\ifLuaTeX
  \usepackage{selnolig}  % disable illegal ligatures
\fi
\usepackage{bookmark}
\IfFileExists{xurl.sty}{\usepackage{xurl}}{} % add URL line breaks if available
\urlstyle{same}
\hypersetup{
  pdftitle={Impact of PM2.5 Exposure on Low Birth Weight in Ulaanbaatar from 2016--2025},
  pdfauthor={Ahmadi, Barua, Bhuyan, Karayel},
  hidelinks,
  pdfcreator={LaTeX via pandoc}}

\title{Impact of PM2.5 Exposure on Low Birth Weight in Ulaanbaatar from
2016--2025}
\author{Ahmadi, Barua, Bhuyan, Karayel}
\date{2025-04-29}

\begin{document}
\maketitle

{
\setcounter{tocdepth}{2}
\tableofcontents
}
\textbf{Rationale and Research Questions} Air pollution, particularly
PM2.5, remains a significant public health concern in Ulaanbaatar.\\
This project examines whether exposure to elevated PM2.5 concentrations
during pregnancy is associated with an increased risk of low birth
weight (LBW) among newborns between 2016 and 2025.

\textbf{Research Questions:} - What are the seasonal and long-term
trends in PM2.5 concentrations in Ulaanbaatar? - Is there a measurable
association between PM2.5 exposure levels and low birth weight rates? -
Does the timing of PM2.5 exposure during pregnancy (examined through
lagged exposure models) influence birth outcomes?

\begin{center}\rule{0.5\linewidth}{0.5pt}\end{center}

\textbf{Dataset Information}

\textbf{Data Sources} \#Birth data consisted of monthly counts of live
births and low birth weight (LBW) births for each district in
Ulaanbaatar. Air pollution data included daily and monthly averages of
PM2.5 concentrations, aggregated to monthly values. These datasets were
merged by month and district for analysis.

\subsection{Data Cleaning and
Wrangling}\label{data-cleaning-and-wrangling}

\begin{itemize}
\tightlist
\item
  \textbf{PM2.5 data}:

  \begin{itemize}
  \tightlist
  \item
    Combined annual CSV files across multiple years.
  \item
    Replaced invalid \texttt{-999} entries with \texttt{NA}.
  \item
    Aggregated hourly readings into daily and monthly averages to align
    with birth record reporting.
  \end{itemize}
\item
  \textbf{Birth outcomes data}:

  \begin{itemize}
  \tightlist
  \item
    Reshaped from wide format to long format for time series analysis.
  \item
    Merged live births and low birth weight counts into a unified
    dataset.
  \end{itemize}
\item
  \textbf{Final merged dataset}:

  \begin{itemize}
  \tightlist
  \item
    Monthly PM2.5 exposure data was linked with corresponding monthly
    birth outcome data by date.
  \end{itemize}
\end{itemize}

\subsection{Dataset Structure}\label{dataset-structure}

\begin{longtable}[]{@{}
  >{\raggedright\arraybackslash}p{(\columnwidth - 4\tabcolsep) * \real{0.2043}}
  >{\raggedright\arraybackslash}p{(\columnwidth - 4\tabcolsep) * \real{0.4301}}
  >{\raggedright\arraybackslash}p{(\columnwidth - 4\tabcolsep) * \real{0.3656}}@{}}
\toprule\noalign{}
\begin{minipage}[b]{\linewidth}\raggedright
Dataset
\end{minipage} & \begin{minipage}[b]{\linewidth}\raggedright
Key Variables
\end{minipage} & \begin{minipage}[b]{\linewidth}\raggedright
Notes
\end{minipage} \\
\midrule\noalign{}
\endhead
\bottomrule\noalign{}
\endlastfoot
PM2.5 Pollution & DateTime, Raw Concentration, AQI & Aggregated to
monthly averages \\
Birth Outcomes & Date, Aimag, Low Birth Weight count, Live Births &
Cleaned and merged records \\
\end{longtable}

\begin{center}\rule{0.5\linewidth}{0.5pt}\end{center}

\begin{Shaded}
\begin{Highlighting}[]
\CommentTok{\#No we load and prepare the raw PM2.5 and birth outcome datasets as described previously.}

\CommentTok{\# Define file paths for PM2.5 data}
\NormalTok{years }\OtherTok{\textless{}{-}} \DecValTok{2015}\SpecialCharTok{:}\DecValTok{2025}
\NormalTok{pm25\_files }\OtherTok{\textless{}{-}} \FunctionTok{paste0}\NormalTok{(}\StringTok{"Data/Raw/Ulaanbaatar\_PM2.5\_"}\NormalTok{, years, }\StringTok{"\_YTD.csv"}\NormalTok{)}
\FunctionTok{names}\NormalTok{(pm25\_files) }\OtherTok{\textless{}{-}}\NormalTok{ years}

\CommentTok{\# Read and bind all PM2.5 files}
\NormalTok{pm25\_all }\OtherTok{\textless{}{-}} \FunctionTok{map\_dfr}\NormalTok{(pm25\_files, }\SpecialCharTok{\textasciitilde{}} \FunctionTok{read\_csv}\NormalTok{(.x, }\AttributeTok{show\_col\_types =} \ConstantTok{FALSE}\NormalTok{))}

\CommentTok{\# Clean and convert {-}999 to NA}
\NormalTok{pm25\_all }\OtherTok{\textless{}{-}}\NormalTok{ pm25\_all }\SpecialCharTok{\%\textgreater{}\%}
  \FunctionTok{mutate}\NormalTok{(}\FunctionTok{across}\NormalTok{(}\FunctionTok{where}\NormalTok{(is.numeric), }\SpecialCharTok{\textasciitilde{}} \FunctionTok{na\_if}\NormalTok{(., }\SpecialCharTok{{-}}\DecValTok{999}\NormalTok{))) }\SpecialCharTok{\%\textgreater{}\%}
  \FunctionTok{clean\_names}\NormalTok{() }\SpecialCharTok{\%\textgreater{}\%}
  \FunctionTok{rename}\NormalTok{(}\AttributeTok{DateTime =}\NormalTok{ date\_lt) }\SpecialCharTok{\%\textgreater{}\%}
  \FunctionTok{mutate}\NormalTok{(}
    \AttributeTok{DateTime =} \FunctionTok{parse\_date\_time}\NormalTok{(DateTime, }\AttributeTok{orders =} \StringTok{"ymd IMp"}\NormalTok{),}
    \AttributeTok{Date     =} \FunctionTok{as\_date}\NormalTok{(DateTime)}
\NormalTok{  )}

\CommentTok{\# Load birth outcome data}
\NormalTok{birth\_weight\_low }\OtherTok{\textless{}{-}} \FunctionTok{read\_csv}\NormalTok{(}\StringTok{"Data/Raw/BIRTH WEIGTH LOWER THAN 2500 GRAMS.csv"}\NormalTok{)}
\end{Highlighting}
\end{Shaded}

\begin{verbatim}
## Rows: 1 Columns: 112
## -- Column specification --------------------------------------------------------
## Delimiter: ","
## chr   (1): Aimag
## dbl (111): 2016-01, 2016-02, 2016-03, 2016-04, 2016-05, 2016-06, 2016-07, 20...
## 
## i Use `spec()` to retrieve the full column specification for this data.
## i Specify the column types or set `show_col_types = FALSE` to quiet this message.
\end{verbatim}

\begin{Shaded}
\begin{Highlighting}[]
\NormalTok{live\_births      }\OtherTok{\textless{}{-}} \FunctionTok{read\_csv}\NormalTok{(}\StringTok{"Data/Raw/LIVE BIRTHS.csv"}\NormalTok{)}
\end{Highlighting}
\end{Shaded}

\begin{verbatim}
## Rows: 1 Columns: 112
## -- Column specification --------------------------------------------------------
## Delimiter: ","
## chr   (1): Aimag
## num (111): 2016-01, 2016-02, 2016-03, 2016-04, 2016-05, 2016-06, 2016-07, 20...
## 
## i Use `spec()` to retrieve the full column specification for this data.
## i Specify the column types or set `show_col_types = FALSE` to quiet this message.
\end{verbatim}

\begin{Shaded}
\begin{Highlighting}[]
\CommentTok{\# Clean numeric columns in live births (in case of commas)}
\ControlFlowTok{for}\NormalTok{ (col }\ControlFlowTok{in} \FunctionTok{names}\NormalTok{(live\_births)[}\SpecialCharTok{{-}}\DecValTok{1}\NormalTok{]) \{}
\NormalTok{  live\_births[[col]] }\OtherTok{\textless{}{-}} \FunctionTok{as.numeric}\NormalTok{(}\FunctionTok{gsub}\NormalTok{(}\StringTok{","}\NormalTok{, }\StringTok{""}\NormalTok{, live\_births[[col]]))}
\NormalTok{\}}

\CommentTok{\# Reshape both datasets to long format}
\NormalTok{birth\_weight\_long }\OtherTok{\textless{}{-}}\NormalTok{ birth\_weight\_low }\SpecialCharTok{\%\textgreater{}\%}
  \FunctionTok{pivot\_longer}\NormalTok{(}\SpecialCharTok{{-}}\NormalTok{Aimag, }\AttributeTok{names\_to =} \StringTok{"Month"}\NormalTok{, }\AttributeTok{values\_to =} \StringTok{"Low\_Birth\_Weight"}\NormalTok{) }\SpecialCharTok{\%\textgreater{}\%}
  \FunctionTok{mutate}\NormalTok{(}\AttributeTok{Month =} \FunctionTok{gsub}\NormalTok{(}\StringTok{"\^{}X"}\NormalTok{, }\StringTok{""}\NormalTok{, Month))}

\NormalTok{live\_births\_long }\OtherTok{\textless{}{-}}\NormalTok{ live\_births }\SpecialCharTok{\%\textgreater{}\%}
  \FunctionTok{pivot\_longer}\NormalTok{(}\SpecialCharTok{{-}}\NormalTok{Aimag, }\AttributeTok{names\_to =} \StringTok{"Month"}\NormalTok{, }\AttributeTok{values\_to =} \StringTok{"Live\_Births"}\NormalTok{) }\SpecialCharTok{\%\textgreater{}\%}
  \FunctionTok{mutate}\NormalTok{(}\AttributeTok{Month =} \FunctionTok{gsub}\NormalTok{(}\StringTok{"\^{}X"}\NormalTok{, }\StringTok{""}\NormalTok{, Month))}

\CommentTok{\# Merge and finalize}
\NormalTok{births\_merged }\OtherTok{\textless{}{-}} \FunctionTok{left\_join}\NormalTok{(birth\_weight\_long, live\_births\_long, }\AttributeTok{by =} \FunctionTok{c}\NormalTok{(}\StringTok{"Aimag"}\NormalTok{, }\StringTok{"Month"}\NormalTok{)) }\SpecialCharTok{\%\textgreater{}\%}
  \FunctionTok{mutate}\NormalTok{(}\AttributeTok{Date =} \FunctionTok{ym}\NormalTok{(Month)) }\SpecialCharTok{\%\textgreater{}\%}
  \FunctionTok{select}\NormalTok{(Aimag, Date, Low\_Birth\_Weight, Live\_Births)}
\end{Highlighting}
\end{Shaded}

\textbf{Exploratory Analysis and Visualizations}

\begin{Shaded}
\begin{Highlighting}[]
\CommentTok{\#We start by analyzing the structure, quality, and time{-}related aspects of the air pollution data.}


\CommentTok{\# Daily average}
\NormalTok{pm25\_daily }\OtherTok{\textless{}{-}}\NormalTok{ pm25\_all }\SpecialCharTok{\%\textgreater{}\%}
  \FunctionTok{group\_by}\NormalTok{(Date) }\SpecialCharTok{\%\textgreater{}\%}
  \FunctionTok{summarize}\NormalTok{(}
    \AttributeTok{raw\_conc\_daily =} \FunctionTok{mean}\NormalTok{(raw\_conc, }\AttributeTok{na.rm =} \ConstantTok{TRUE}\NormalTok{),}
    \AttributeTok{aqi\_daily =} \FunctionTok{mean}\NormalTok{(aqi, }\AttributeTok{na.rm =} \ConstantTok{TRUE}\NormalTok{),}
    \AttributeTok{.groups =} \StringTok{"drop"}
\NormalTok{  )}

\CommentTok{\# Monthly average}
\NormalTok{pm25\_monthly }\OtherTok{\textless{}{-}}\NormalTok{ pm25\_daily }\SpecialCharTok{\%\textgreater{}\%}
  \FunctionTok{mutate}\NormalTok{(}\AttributeTok{Month =} \FunctionTok{floor\_date}\NormalTok{(Date, }\StringTok{"month"}\NormalTok{)) }\SpecialCharTok{\%\textgreater{}\%}
  \FunctionTok{group\_by}\NormalTok{(Month) }\SpecialCharTok{\%\textgreater{}\%}
  \FunctionTok{summarize}\NormalTok{(}
    \AttributeTok{raw\_conc\_monthly =} \FunctionTok{mean}\NormalTok{(raw\_conc\_daily, }\AttributeTok{na.rm =} \ConstantTok{TRUE}\NormalTok{),}
    \AttributeTok{aqi\_monthly =} \FunctionTok{mean}\NormalTok{(aqi\_daily, }\AttributeTok{na.rm =} \ConstantTok{TRUE}\NormalTok{),}
    \AttributeTok{.groups =} \StringTok{"drop"}
\NormalTok{  )}
\end{Highlighting}
\end{Shaded}

\begin{Shaded}
\begin{Highlighting}[]
\FunctionTok{ggplot}\NormalTok{(pm25\_daily, }\FunctionTok{aes}\NormalTok{(}\AttributeTok{x =}\NormalTok{ Date, }\AttributeTok{y =}\NormalTok{ raw\_conc\_daily)) }\SpecialCharTok{+}
  \FunctionTok{geom\_line}\NormalTok{(}\AttributeTok{color =} \StringTok{"steelblue"}\NormalTok{) }\SpecialCharTok{+}
  \FunctionTok{labs}\NormalTok{(}
    \AttributeTok{title =} \StringTok{"Daily PM2.5 Concentrations in Ulaanbaatar (µg/m³)"}\NormalTok{,}
    \AttributeTok{x =} \StringTok{"Date"}\NormalTok{,}
    \AttributeTok{y =} \StringTok{"Daily Mean PM2.5"}
\NormalTok{  ) }\SpecialCharTok{+}
  \FunctionTok{theme\_minimal}\NormalTok{()}
\end{Highlighting}
\end{Shaded}

\begin{verbatim}
## Warning: Removed 305 rows containing missing values or values outside the scale range
## (`geom_line()`).
\end{verbatim}

\includegraphics{Final-Project_files/figure-latex/PM2.5 Trends Over Time-1.pdf}

\#The plot titled ``Daily PM2.5 Concentrations in Ulaanbaatar (µg/m³)''
shows a time series of daily mean PM2.5 levels from 2015 through early
2025. The data reveals a recurring pattern of sharp spikes in pollution
levels each year, with the highest concentrations consistently occurring
during the colder months. These peaks reach over 600 µg/m³ in some
years, notably in 2016 and 2017, suggesting severe air quality events.
Although the magnitude of these spikes varies year to year, the pattern
of elevated concentrations in specific periods remains consistent
throughout the decade, indicating persistent seasonal pollution events.

\begin{verbatim}


``` r
#We evaluate missingness by calculating how many months each year had incomplete PM2.5 daily records.

pm25_yearly_missing <- pm25_monthly %>%
  mutate(Year = year(Month)) %>%
  group_by(Year) %>%
  summarize(
    months_with_missing = sum(is.na(raw_conc_monthly)),
    .groups = "drop"
  )

ggplot(pm25_yearly_missing, aes(x = Year, y = months_with_missing)) +
  geom_col(fill = "tomato") +
  labs(
    title = "Figure 2. Number of Months with Missing PM2.5 Data",
    x = "Year",
    y = "Months with Missing PM2.5"
  ) +
  theme_minimal()
\end{verbatim}

\includegraphics{Final-Project_files/figure-latex/Missing Data Patterns in PM2.5-1.pdf}

\#Our analysis in Figure 2 reveals that data gaps were mostly depicted
in 2016, with 10 months lacking complete PM2.5 measurements.
Additionally, one month in 2017 also shows missing data. From 2018
onward, there have been no recorded months with missing data. Therefore,
the team had to analyse months with missing PM2.5 data

\begin{verbatim}


``` r
#We also explore the variability of monthly PM2.5 to check for seasonal peaks and outliers.

ggplot(pm25_monthly, aes(y = raw_conc_monthly)) +
  geom_boxplot(outlier.color = "red") +
  labs(
    title = "Figure 3. Distribution of Monthly PM2.5 Concentrations",
    y = "Monthly Mean PM2.5 (µg/m³)"
  ) +
  theme_minimal()
\end{verbatim}

\begin{verbatim}
## Warning: Removed 11 rows containing non-finite outside the scale range
## (`stat_boxplot()`).
\end{verbatim}

\includegraphics{Final-Project_files/figure-latex/Distribution of Monthly PM2.5 Concentrations-1.pdf}

\#The boxplot summarizes the variation in monthly mean PM2.5 levels in
Ulaanbaatar. The plot illustrates that the median monthly PM2.5
concentration is just above 50 µg/m³, with the interquartile range
(IQR)---representing the middle 50\% of values---extending from
approximately 20 to 110 µg/m³. Several outliers are displayed above the
upper whisker, with some months reporting concentrations exceeding 300
µg/m³. These outliers likely indicate extreme pollution events occurring
in particular months, aligning with the sharp peaks in Figure 1. The
extended upper whisker signifies a right-skewed distribution, implying
that while high PM2.5 values are less common, they can be significantly
elevated when they do occur. In summary, this boxplot demonstrates that
most monthly averages remain substantially above the WHO 24-hour
guideline of 15 µg/m³, with a few months experiencing extremely high
levels, underscoring the seriousness and fluctuations of air pollution
in Ulaanbaatar.

\begin{verbatim}



``` r
#Before merging exposure and outcomes, we visualize trends in live births and low birth weight births separately.

# Plot live births over time
ggplot(births_merged, aes(x = Date, y = Live_Births)) +
  geom_line(color = "darkgreen") +
  labs(
    title = "Figure 4. Monthly Live Births in Ulaanbaatar",
    x = "Date",
    y = "Number of Live Births"
  ) +
  theme_minimal()
\end{verbatim}

\includegraphics{Final-Project_files/figure-latex/Birth Outcomes Over Time-1.pdf}

\begin{Shaded}
\begin{Highlighting}[]
\CommentTok{\# Plot low birth weight births over time}
\FunctionTok{ggplot}\NormalTok{(births\_merged, }\FunctionTok{aes}\NormalTok{(}\AttributeTok{x =}\NormalTok{ Date, }\AttributeTok{y =}\NormalTok{ Low\_Birth\_Weight)) }\SpecialCharTok{+}
  \FunctionTok{geom\_line}\NormalTok{(}\AttributeTok{color =} \StringTok{"purple"}\NormalTok{) }\SpecialCharTok{+}
  \FunctionTok{labs}\NormalTok{(}
    \AttributeTok{title =} \StringTok{"Figure 5. Monthly Low Birth Weight Births in Ulaanbaatar"}\NormalTok{,}
    \AttributeTok{x =} \StringTok{"Date"}\NormalTok{,}
    \AttributeTok{y =} \StringTok{"Number of LBW Births"}
\NormalTok{  ) }\SpecialCharTok{+}
  \FunctionTok{theme\_minimal}\NormalTok{()}
\end{Highlighting}
\end{Shaded}

\includegraphics{Final-Project_files/figure-latex/Birth Outcomes Over Time-2.pdf}

\begin{Shaded}
\begin{Highlighting}[]
\NormalTok{\#This chart shows that between 2016 and approximately 2021, the rate of LBW births was notably high and fluctuated significantly, consistently surpassing 160 births per month, occasionally exceeding 200. However, starting in 2022, a clear downward trend is evident, with both the average rate of LBW births and their levels of variability decreasing. By 2024, the number of LBW births often dips below 150 per month, marking some of the lowest figures recorded during the observed timeframe.}
\end{Highlighting}
\end{Shaded}

\#Merge Exposure and Outcome Data We link monthly PM2.5 exposure data
with corresponding monthly birth outcomes. Therefore, in this section,
we merge exposure and outcome datasets, calculate the low birth weight
(LBW) rate, and begin preliminary modeling to investigate associations
between PM2.5 exposure and birth outcomes.

\begin{Shaded}
\begin{Highlighting}[]
\CommentTok{\# Merge datasets by Date}
\NormalTok{full\_data }\OtherTok{\textless{}{-}}\NormalTok{ births\_merged }\SpecialCharTok{\%\textgreater{}\%}
  \FunctionTok{left\_join}\NormalTok{(pm25\_monthly, }\AttributeTok{by =} \FunctionTok{c}\NormalTok{(}\StringTok{"Date"} \OtherTok{=} \StringTok{"Month"}\NormalTok{)) }\SpecialCharTok{\%\textgreater{}\%}
  \FunctionTok{arrange}\NormalTok{(Date)}
\end{Highlighting}
\end{Shaded}

\#Calculate Low Birth Weight Rate \# here we create a new variable:

LBW\_rate = (Low Birth Weight / Live Births) * 100

\begin{Shaded}
\begin{Highlighting}[]
\CommentTok{\# Calculate LBW rate as percentage}
\NormalTok{full\_data }\OtherTok{\textless{}{-}}\NormalTok{ full\_data }\SpecialCharTok{\%\textgreater{}\%}
  \FunctionTok{mutate}\NormalTok{(}\AttributeTok{LBW\_rate =} \DecValTok{100} \SpecialCharTok{*}\NormalTok{ Low\_Birth\_Weight }\SpecialCharTok{/}\NormalTok{ Live\_Births)}
\end{Highlighting}
\end{Shaded}

\#Summary Statistics

\begin{Shaded}
\begin{Highlighting}[]
\NormalTok{summary\_stats }\OtherTok{\textless{}{-}}\NormalTok{ full\_data }\SpecialCharTok{\%\textgreater{}\%}
  \FunctionTok{summarise}\NormalTok{(}
    \AttributeTok{Mean\_LBW\_Rate =} \FunctionTok{mean}\NormalTok{(LBW\_rate, }\AttributeTok{na.rm =} \ConstantTok{TRUE}\NormalTok{),}
    \AttributeTok{Median\_LBW\_Rate =} \FunctionTok{median}\NormalTok{(LBW\_rate, }\AttributeTok{na.rm =} \ConstantTok{TRUE}\NormalTok{),}
    \AttributeTok{Mean\_PM25 =} \FunctionTok{mean}\NormalTok{(raw\_conc\_monthly, }\AttributeTok{na.rm =} \ConstantTok{TRUE}\NormalTok{),}
    \AttributeTok{Median\_PM25 =} \FunctionTok{median}\NormalTok{(raw\_conc\_monthly, }\AttributeTok{na.rm =} \ConstantTok{TRUE}\NormalTok{)}
\NormalTok{  )}

\NormalTok{summary\_stats }\SpecialCharTok{\%\textgreater{}\%}
  \FunctionTok{kable}\NormalTok{(}\AttributeTok{caption =} \StringTok{"Table 1. Summary Statistics of LBW Rate and PM2.5"}\NormalTok{, }\AttributeTok{digits =} \DecValTok{2}\NormalTok{) }\SpecialCharTok{\%\textgreater{}\%}
  \FunctionTok{kable\_styling}\NormalTok{(}\AttributeTok{full\_width =} \ConstantTok{FALSE}\NormalTok{)}
\end{Highlighting}
\end{Shaded}

\begin{longtable}[t]{rrrr}
\caption{\label{tab:unnamed-chunk-3}Table 1. Summary Statistics of LBW Rate and PM2.5}\\
\toprule
Mean\_LBW\_Rate & Median\_LBW\_Rate & Mean\_PM25 & Median\_PM25\\
\midrule
5.03 & 4.96 & 67.28 & 35.22\\
\bottomrule
\end{longtable}

\#Scatterplot: PM2.5 vs LBW Rate \# Now we visualize the bivariate
relationship before modeling.

\begin{Shaded}
\begin{Highlighting}[]
\FunctionTok{ggplot}\NormalTok{(full\_data, }\FunctionTok{aes}\NormalTok{(}\AttributeTok{x =}\NormalTok{ raw\_conc\_monthly, }\AttributeTok{y =}\NormalTok{ LBW\_rate)) }\SpecialCharTok{+}
  \FunctionTok{geom\_point}\NormalTok{(}\AttributeTok{alpha =} \FloatTok{0.7}\NormalTok{) }\SpecialCharTok{+}
  \FunctionTok{geom\_smooth}\NormalTok{(}\AttributeTok{method =} \StringTok{"lm"}\NormalTok{, }\AttributeTok{se =} \ConstantTok{TRUE}\NormalTok{, }\AttributeTok{color =} \StringTok{"blue"}\NormalTok{) }\SpecialCharTok{+}
  \FunctionTok{labs}\NormalTok{(}
    \AttributeTok{title =} \StringTok{"Figure 6. Relationship between Monthly PM2.5 and LBW Rate"}\NormalTok{,}
    \AttributeTok{x =} \StringTok{"Monthly PM2.5 (µg/m³)"}\NormalTok{,}
    \AttributeTok{y =} \StringTok{"Low Birth Weight Rate (\%)"}
\NormalTok{  ) }\SpecialCharTok{+}
  \FunctionTok{theme\_minimal}\NormalTok{()}
\end{Highlighting}
\end{Shaded}

\begin{verbatim}
## `geom_smooth()` using formula = 'y ~ x'
\end{verbatim}

\begin{verbatim}
## Warning: Removed 4 rows containing non-finite outside the scale range
## (`stat_smooth()`).
\end{verbatim}

\begin{verbatim}
## Warning: Removed 4 rows containing missing values or values outside the scale range
## (`geom_point()`).
\end{verbatim}

\includegraphics{Final-Project_files/figure-latex/unnamed-chunk-4-1.pdf}
\#The plot reveals a positive trend, indicating that higher levels of
PM2.5 are associated with slightly higher rates of low birth weight.
Although the slope of the regression line is not steep, the upward
direction suggests that increases in air pollution may contribute to
adverse birth outcomes. The shaded area around the line represents the
95\% confidence interval, showing some uncertainty in the estimate, but
the trend remains visible.

\begin{verbatim}

#Simple Linear Regression
#We fit an initial simple model:
LBW Rate ~ Monthly PM2.5 Concentration


``` r
model_simple <- lm(LBW_rate ~ raw_conc_monthly, data = full_data)

tidy(model_simple) %>%
  kable(caption = "Table 2. Simple Linear Regression of LBW Rate on PM2.5", digits = 3) %>%
  kable_styling(full_width = FALSE)
\end{verbatim}

\begin{longtable}[t]{lrrrr}
\caption{\label{tab:unnamed-chunk-5}Table 2. Simple Linear Regression of LBW Rate on PM2.5}\\
\toprule
term & estimate & std.error & statistic & p.value\\
\midrule
(Intercept) & 4.928 & 0.072 & 68.089 & 0.000\\
raw\_conc\_monthly & 0.001 & 0.001 & 1.597 & 0.113\\
\bottomrule
\end{longtable}

\#The plot shows a positive trend, indicating that higher levels of
PM2.5 are associated with slightly higher rates of low birth weight.
Although the slope of the regression line is not steep, the upward
direction suggests that increases in air pollution may contribute to
adverse birth outcomes. The shaded area around the line represents the
95\% confidence interval, showing some uncertainty in the estimate, but
the trend remains visible.

\begin{verbatim}
\end{verbatim}

\#Lagged Exposure Variables We construct lagged PM2.5 variables to
capture exposures during previous months relative to the birth month.

\begin{Shaded}
\begin{Highlighting}[]
\NormalTok{full\_data }\OtherTok{\textless{}{-}}\NormalTok{ full\_data }\SpecialCharTok{\%\textgreater{}\%}
  \FunctionTok{arrange}\NormalTok{(Date) }\SpecialCharTok{\%\textgreater{}\%}
  \FunctionTok{mutate}\NormalTok{(}
    \AttributeTok{pm25\_lag0 =}\NormalTok{ raw\_conc\_monthly,            }\CommentTok{\# Current month}
    \AttributeTok{pm25\_lag1 =} \FunctionTok{lag}\NormalTok{(raw\_conc\_monthly, }\DecValTok{1}\NormalTok{),     }\CommentTok{\# 1 month before}
    \AttributeTok{pm25\_lag2 =} \FunctionTok{lag}\NormalTok{(raw\_conc\_monthly, }\DecValTok{2}\NormalTok{),     }\CommentTok{\# 2 months before}
    \AttributeTok{pm25\_lag3 =} \FunctionTok{lag}\NormalTok{(raw\_conc\_monthly, }\DecValTok{3}\NormalTok{)      }\CommentTok{\# 3 months before}
\NormalTok{  )}
\end{Highlighting}
\end{Shaded}

\begin{Shaded}
\begin{Highlighting}[]
\CommentTok{\# Now We fit a distributed lag linear model:}
\CommentTok{\#LBW Rate \textasciitilde{} PM2.5 exposure in current and previous 3 months}

\NormalTok{model\_lag }\OtherTok{\textless{}{-}} \FunctionTok{lm}\NormalTok{(LBW\_rate }\SpecialCharTok{\textasciitilde{}}\NormalTok{ pm25\_lag0 }\SpecialCharTok{+}\NormalTok{ pm25\_lag1 }\SpecialCharTok{+}\NormalTok{ pm25\_lag2 }\SpecialCharTok{+}\NormalTok{ pm25\_lag3, }\AttributeTok{data =}\NormalTok{ full\_data)}

\FunctionTok{tidy}\NormalTok{(model\_lag) }\SpecialCharTok{\%\textgreater{}\%}
  \FunctionTok{kable}\NormalTok{(}\AttributeTok{caption =} \StringTok{"Table 3. Distributed Lag Model of LBW Rate on PM2.5 Exposure"}\NormalTok{, }\AttributeTok{digits =} \DecValTok{3}\NormalTok{) }\SpecialCharTok{\%\textgreater{}\%}
  \FunctionTok{kable\_styling}\NormalTok{(}\AttributeTok{full\_width =} \ConstantTok{FALSE}\NormalTok{)}
\end{Highlighting}
\end{Shaded}

\begin{longtable}[t]{lrrrr}
\caption{\label{tab:Distributed Lag Model}Table 3. Distributed Lag Model of LBW Rate on PM2.5 Exposure}\\
\toprule
term & estimate & std.error & statistic & p.value\\
\midrule
(Intercept) & 5.053 & 0.097 & 52.155 & 0.000\\
pm25\_lag0 & -0.001 & 0.002 & -0.369 & 0.713\\
pm25\_lag1 & 0.002 & 0.003 & 0.750 & 0.455\\
pm25\_lag2 & 0.001 & 0.003 & 0.441 & 0.660\\
pm25\_lag3 & -0.003 & 0.002 & -1.742 & 0.085\\
\bottomrule
\end{longtable}

\#The distributed lag model does not find statistically significant
evidence that monthly PM2.5 exposure from 0 to 3 months prior has a
measurable impact on LBW rate in this dataset. However, the
near-significant result for lag 3 could indicate a potential delayed
effect worth further investigation, perhaps with more data or a
different model specification.

\begin{verbatim}


``` r
#Cumulative Exposure Variables: We also create cumulative exposure averages to smooth over lag periods, as fetal development may be affected by sustained pollution rather than isolated monthly spikes.

full_data <- full_data %>%
  mutate(
    pm25_cum12 = (pm25_lag1 + pm25_lag2) / 2,          # 1–2 months average
    pm25_cum123 = (pm25_lag1 + pm25_lag2 + pm25_lag3) / 3  # 1–3 months average
  )
  
#Analysis: Here, we also get the same result we got for the distributed lag model
\end{verbatim}

\begin{Shaded}
\begin{Highlighting}[]
\CommentTok{\#Finally, we fit simple models of LBW Rate on cumulative exposures:}

\NormalTok{model\_cum12 }\OtherTok{\textless{}{-}} \FunctionTok{lm}\NormalTok{(LBW\_rate }\SpecialCharTok{\textasciitilde{}}\NormalTok{ pm25\_cum12, }\AttributeTok{data =}\NormalTok{ full\_data)}
\NormalTok{model\_cum123 }\OtherTok{\textless{}{-}} \FunctionTok{lm}\NormalTok{(LBW\_rate }\SpecialCharTok{\textasciitilde{}}\NormalTok{ pm25\_cum123, }\AttributeTok{data =}\NormalTok{ full\_data)}

\FunctionTok{bind\_rows}\NormalTok{(}
  \FunctionTok{tidy}\NormalTok{(model\_cum12) }\SpecialCharTok{\%\textgreater{}\%} \FunctionTok{mutate}\NormalTok{(}\AttributeTok{Model =} \StringTok{"Cumulative Lag 1{-}2"}\NormalTok{),}
  \FunctionTok{tidy}\NormalTok{(model\_cum123) }\SpecialCharTok{\%\textgreater{}\%} \FunctionTok{mutate}\NormalTok{(}\AttributeTok{Model =} \StringTok{"Cumulative Lag 1{-}3"}\NormalTok{)}
\NormalTok{) }\SpecialCharTok{\%\textgreater{}\%}
  \FunctionTok{select}\NormalTok{(Model, term, estimate, std.error, statistic, p.value) }\SpecialCharTok{\%\textgreater{}\%}
  \FunctionTok{kable}\NormalTok{(}\AttributeTok{caption =} \StringTok{"Table 4. Regression of LBW Rate on Cumulative PM2.5 Exposures"}\NormalTok{, }\AttributeTok{digits =} \DecValTok{3}\NormalTok{) }\SpecialCharTok{\%\textgreater{}\%}
  \FunctionTok{kable\_styling}\NormalTok{(}\AttributeTok{full\_width =} \ConstantTok{FALSE}\NormalTok{)}
\end{Highlighting}
\end{Shaded}

\begin{longtable}[t]{llrrrr}
\caption{\label{tab:Regression on Cumulative Exposure}Table 4. Regression of LBW Rate on Cumulative PM2.5 Exposures}\\
\toprule
Model & term & estimate & std.error & statistic & p.value\\
\midrule
Cumulative Lag 1-2 & (Intercept) & 4.944 & 0.080 & 62.183 & 0.000\\
Cumulative Lag 1-2 & pm25\_cum12 & 0.001 & 0.001 & 1.421 & 0.158\\
Cumulative Lag 1-3 & (Intercept) & 4.985 & 0.085 & 58.483 & 0.000\\
Cumulative Lag 1-3 & pm25\_cum123 & 0.001 & 0.001 & 0.863 & 0.390\\
\bottomrule
\end{longtable}

\begin{Shaded}
\begin{Highlighting}[]
\CommentTok{\#Analysis:Both cumulative models show a very small estimated increase in LBW rate (0.001 percentage points) per unit increase in cumulative PM2.5 exposure, but neither result is statistically significant. This suggests that, based on the available data, cumulative PM2.5 exposure in the last 2–3 months of pregnancy is not strongly or consistently associated with changes in LBW rate.}
\end{Highlighting}
\end{Shaded}

\begin{Shaded}
\begin{Highlighting}[]
\CommentTok{\#Now we assess multicollinearity among the lagged PM2.5 variables using VIFs.A VIF greater than 5–10 indicates problematic multicollinearity.}

\CommentTok{\# Load car package if not already}
\FunctionTok{library}\NormalTok{(car)}

\CommentTok{\# Fit lagged model again if needed}
\NormalTok{model\_lag }\OtherTok{\textless{}{-}} \FunctionTok{lm}\NormalTok{(LBW\_rate }\SpecialCharTok{\textasciitilde{}}\NormalTok{ pm25\_lag0 }\SpecialCharTok{+}\NormalTok{ pm25\_lag1 }\SpecialCharTok{+}\NormalTok{ pm25\_lag2 }\SpecialCharTok{+}\NormalTok{ pm25\_lag3, }\AttributeTok{data =}\NormalTok{ full\_data)}

\CommentTok{\# Calculate VIFs}
\NormalTok{vif\_values }\OtherTok{\textless{}{-}} \FunctionTok{vif}\NormalTok{(model\_lag)}

\CommentTok{\# Display neatly}
\NormalTok{vif\_values }\SpecialCharTok{\%\textgreater{}\%}
  \FunctionTok{as.data.frame}\NormalTok{() }\SpecialCharTok{\%\textgreater{}\%}
  \FunctionTok{rownames\_to\_column}\NormalTok{(}\StringTok{"Predictor"}\NormalTok{) }\SpecialCharTok{\%\textgreater{}\%}
  \FunctionTok{rename}\NormalTok{(}\AttributeTok{VIF =} \StringTok{"."}\NormalTok{) }\SpecialCharTok{\%\textgreater{}\%}
  \FunctionTok{kable}\NormalTok{(}\AttributeTok{caption =} \StringTok{"Table 5. Variance Inflation Factors for Lagged PM2.5 Model"}\NormalTok{, }\AttributeTok{digits =} \DecValTok{2}\NormalTok{) }\SpecialCharTok{\%\textgreater{}\%}
  \FunctionTok{kable\_styling}\NormalTok{(}\AttributeTok{full\_width =} \ConstantTok{FALSE}\NormalTok{)}
\end{Highlighting}
\end{Shaded}

\begin{longtable}[t]{lr}
\caption{\label{tab:Multicollinearity Diagnostics}Table 5. Variance Inflation Factors for Lagged PM2.5 Model}\\
\toprule
Predictor & VIF\\
\midrule
pm25\_lag0 & 4.06\\
pm25\_lag1 & 8.89\\
pm25\_lag2 & 8.08\\
pm25\_lag3 & 3.44\\
\bottomrule
\end{longtable}

\begin{Shaded}
\begin{Highlighting}[]
\CommentTok{\#Analysis:The high VIF values (especially for lags 1 and 2) suggest that the lagged PM2.5 variables are strongly correlated with one another, which can inflate standard errors and make it difficult to detect statistically significant effects in your model.}
\end{Highlighting}
\end{Shaded}

\textbf{Methodology and Limitations}

\#Statistical Approach and Results \textbf{The analysis began with a
simple linear regression to test the association between monthly PM2.5
concentrations and LBW rates in Ulaanbaatar. This initial model, which
regressed LBW rates on same-month PM2.5 levels, produced a weak and
statistically non-significant relationship {[}p = 0.113{]}.}

\textbf{To account for the possibility that air pollution during
pregnancy may affect birth outcomes with a delay, a distributed lag
model was applied, incorporating PM2.5 exposures lagged by 0--3 months.
However, this approach revealed high multicollinearity among lagged
variables (variance inflation factors greater than 8 for some lags),
which inflated standard errors and made it difficult to isolate the
effect of any single month's exposure.}

\textbf{Cumulative exposure models were then constructed by averaging
PM2.5 over biologically plausible windows (e.g., 1--2 months and 1--3
months before birth) to capture sustained exposure during critical
gestational periods. Despite this, the results remained non-significant,
with small coefficients beta = 0.001, high p values {[}p = 0.158 for 1-2
months, p = 0.390 for 1-3 months{]}}

\textbf{These findings contrast with robust associations reported in the
literature, where studies using individual-level data and precise
gestational timing (such as aligning PM2.5 exposure with the second or
third trimester) consistently found significant effects (Amnuaylojaroen
\& Saokaew, 2024; Zhang et al., 2019). For example, multinational
meta-analyses and cohort studies that adjusted for confounders like
socioeconomic status and healthcare access have reported a 5--8\%
increase in LBW risk per 10 µg/m³ rise in PM2.5 (Lee \& Holm, 2022). The
lack of significance in the current analysis likely stems from
methodological limitations, including the use of aggregated monthly data
(which cannot pinpoint critical gestational windows), unmeasured
confounders (such as maternal age and poverty), and ecological bias.}

\textbf{Discussion} \#The present analysis did not identify a
statistically significant relationship between monthly PM2.5
concentrations and LBW rates in Ulaanbaatar, regardless of whether
exposure was considered in the same month, in previous months, or as a
cumulative average. This outcome contrasts with findings from other
regions, where higher air pollution during pregnancy has been linked to
increased risk of LBW, especially in low-income or high-exposure
settings (Amnuaylojaroen \& Saokaew, 2024; Lee \& Holm, 2022; Zhang et
al., 2019). The discrepancy likely reflects differences in data
structure and methodology. Specifically, the use of monthly aggregated
data in this study limited the ability to align pollution exposure
precisely with the most sensitive periods of pregnancy. Additionally,
the absence of individual-level data and adjustment for important
confounders, such as socioeconomic status and access to healthcare, may
have masked subtle or time-specific effects of PM2.5 exposure.

\#Limitations Several limitations should be considered. First, monthly
aggregated data limited the ability to pinpoint exposure during the most
critical weeks of pregnancy, which is essential in air pollution and
birth outcome research (Amnuaylojaroen \& Saokaew, 2024; Zhang et al.,
2019). Second, important confounders such as maternal age, income, and
access to healthcare were not included, even though these factors are
known to influence both pollution exposure and birth outcomes (Lee \&
Holm, 2022). Third, reliance on group-level data rather than individual
pregnancy records may have introduced ecological bias. Fourth, high
correlation between PM2.5 levels in adjacent months made it difficult to
determine which period of exposure had the strongest effect. Finally,
the sample size and time span may not have been sufficient to detect
small or modest effects. Future research should use data that allow for
more accurate timing of exposure and include additional variables to
better understand how air pollution affects birth outcomes in Mongolia.

\textbf{Key findings} - PM2.5 concentrations in Ulaanbaatar showed
strong seasonal variation, with extreme peaks in winter months. - The
highest pollution levels occurred in 2016 and 2017, exceeding 600 µg/m³
on some days. - Monthly low birth weight (LBW) rates showed a declining
trend starting in 2022. - Simple and lagged linear models did not detect
a statistically significant association between PM2.5 exposure and LBW
rate. - Multicollinearity among lagged exposure variables was high,
limiting the interpretability of individual lag effects.

\textbf{conclusion} While elevated PM2.5 exposure in Ulaanbaatar is
clearly a recurring and serious public health concern, this analysis
does not find strong statistical evidence that short-term monthly or
lagged PM2.5 exposure is independently associated with changes in LBW
rates. Future studies should explore longer exposure windows and
consider individual-level birth data to strengthen causal inference as
we lacked the birth data here.

\textbf{References} Amnuaylojaroen, T., \& Saokaew, S. (2024). Prenatal
PM2.5 exposure and its association with low birth weight: A systematic
review and meta-analysis. Toxics, 12, 446. Lee, J. R., \& Holm, S. M.
(2022). The association between ambient PM2.5 and low birth weight in
California. International Journal of Environmental Research and Public
Health, 19, 13554. Zhang, Y., Wang, J., Chen, L., et al.~(2019). Ambient
PM2.5 and clinically recognized early pregnancy loss: A case-control
study with spatiotemporal exposure predictions. Environment
International, 126, 422--429.

\end{document}
